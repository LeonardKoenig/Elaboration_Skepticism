 \documentclass[12pt,a4paper]{article}
\usepackage[utf8]{inputenc}
\usepackage[ngerman]{babel}
\usepackage[T1]{fontenc}
\usepackage{csquotes}

\usepackage[left=2cm,right=2cm,top=2cm,bottom=2cm]{geometry}

\begin{document}


\textbf{Sokrates}
\begin{itemize}
\item verschiedene Auslegungen:
\begin{itemize}
\item Devise des sokr./methd. Zweifels
\item Skeptisch
\end{itemize}
\end{itemize}

----------------------

Skeptizismus:\\
\textbf{Duden:} \glqq Im weiteren Sinn philosophische Positionen, die Wahrheitsansprüchen gegenüber Verzicht und Zurückhaltung üben und sorgfältige kritische Prüfung verlangen.\\
Im strengeren Sinne Richtungen, die die Beweisbarkeit von Wahrheit (entweder prinzipiell oder partiell) in Zweifel ziehen.\grqq\\
\ \\
Beispiel für einen modernen Skeptiker: David Hume

----------------------

\textbf{Agnostizismus:}
\begin{itemize}
\item Thomas Henry Huxley
\item[$\Rightarrow$] Unabhängigkeit der Wissenschaft von Religion; nur Überirdisches anzweifeln
\item Etym.: \\
a-gnosis: \glqq ohne Wissen\grqq\
\item[$\Rightarrow$] vereinzelt synonym zu Skeptizismus
\end{itemize}

------------------------

\textbf{Solipsismus:}
\begin{itemize}
\item Skepsis bzgl. allem | sich selbst
\item nur eigene Existenz sicher
\item od.: Abhängig vom Zustand des (denkenden) Ichs
\end{itemize}

-----------------------

\textbf{hist. Kontext:}\\
Sophisten = Wanderlehrer:
\begin{itemize}
\item Diskussion
\item Politik
\item Geschichte
\item \emph{Philosophie \& Erkenntnistheorie}
\end{itemize}
zeitbedingt: viele unterschiedliche Schulen

---------------------

Gorgias: dogmatischer Skeptizismus:
\begin{itemize}
\item Nichts existiert;
\item Wenn etwas existiert, dann kann man nichts darüber wissen; und
\item Wenn man etwas darüber wissen kann, dann kann dieses nicht mitgeteilt werden
\end{itemize}

--------------------

\textbf{Radikaler Skeptizismus:}\\
$\rightarrow$ kritische Ansichten gegenüber \emph{jeder} Aussage:\\
\glqq Sinneserfahrungen und Ansichten  weder wahr noch falsch\grqq\ \\
\textbf{Akademischer Skeptizismus:}\\
\begin{itemize}
\item Eingeläutet durch Arkesilaos in der Akademie
\item[$\rightarrow$] keinerlei Wissen möglich
\end{itemize}

------------------

\textbf{Augustinus:}\\
\begin{itemize}
\item Cicero: akadem. Skeptizismus
\item \glqq Römerbriefe\grqq : Bekehrung
\item Prägte platonische Auslegung der Bibel
\end{itemize}
\glqq Wird jemand darüber zweifeln, dass er lebt, sich erinnert, Einsichten hat, will, denkt, weiß und urteilt? [\ldots] Mag einer auch sonst zweifeln, über was er will, über diese Zweifel selbst kann er nicht zweifeln\grqq\

-----------------------

\textbf{Descartes:}
\begin{itemize}
\item  methodischen Zweifel +  \glqq ich zweifle, also bin ich\grqq
\item Durch reine Logik Erkenntnis
\item Vor.: man zweifelt am Zweifeln => Zweifelt man => existiert man
\end{itemize}
\emph{Doch: Weiß man, dass man wirklich zweifelt?}

------------------------


\textbf{David Hume:}
\begin{itemize}
\item Empiriker: Erkenntnisgewinn nur durch Erfahrung
\item[$\Rightarrow$] es liegt nahe, dass unsere Erfahrung getäuscht ist
\item[$\Rightarrow$] kein \glqq echter\grqq\ Erkenntnisgewinn möglich?
\end{itemize}
\ \\
$\Rightarrow$ Was ist überhaupt \glqq Erkenntnis\grqq ?\\
Solche und andere Fragen erläutert er in seinem Werk  \glqq A Treatise of Human Nature\grqq


------------------------


\textbf{Immanuel Kant:}\\
Zu einer  \glqq gereiften [\ldots] Urteilskraft\grqq\ gehöre es auch, dass diese \glqq feste und ihrer Allgemeinheit nach bewährte Maximen zum Grunde hat\grqq\
\begin{itemize}
\item[$\Rightarrow$] wichtig: gg. Dogmatismus
\item[$\Rightarrow$] moralisch Falsch; nicht zur Rechtfertigung einer Lethargie:\\
\glqq Ich kann ja sowieso nichts wissen\grqq
\end{itemize}

-------------------------

\textbf{Ethik:}
\begin{itemize}
\item ethisches Terrain
\item[$\rightarrow$] Nur Begründung, keine Rechtfertigung?
\item möglicherweise $\exists$ doch eine Welt? Schuld eines Beitrags?
\end{itemize}
Nietzsche:\\
\begin{itemize}
\item Verlust der Metaphysik durch unendlichen Zweifel
\item[$\Rightarrow$] kein Fundament der Moral mehr gegeben
\item[$\Rightarrow$] Nihilismus ist eine Entwertung der obersten Werte der Moral
\item \glqq Umwertung aller Werte\grqq\ als neue Basis für die Ethik
\item Nelson: Gesellschaftswissenschaft + Politik: \glqq Die Unmöglichkeit der Erkenntnistheorie\grqq\ 
\end{itemize}

---------------------

\textbf{Popper:}
\begin{itemize}
\item praktisch angewandter Skeptizismus:
\item[$\Rightarrow$] Kritik an Induktion u.Ä.
\item jedoch glaube an wissenschaftlich korrekter Forschung
\end{itemize}
$\rightarrow$ \glqq Kritischer Realismus\grqq


---------------------

%\section{Reflexion – Skeptizismus als Lebenseinstellung?}
\textbf{Fragen:}
\begin{itemize}
\item Sicht im Kontext: Leitfrage?\\
verschiedene Haltungen: Differenzieren?
\item Konsistenz?
\item epistemologischer Sinn?\\
sinnvoll in abstrakterem Kontext?
\end{itemize}


------------------------

erkenntnistheoretische + metaphysische Aspekte relevant:\\
$\Rightarrow$ Radikaler Skeptizismus\\
\ \\
\glqq triviale\grqq , logische Bearbeitung eines komplexen Themas\\
$\rightarrow$ trotzdem keine \glqq Flucht\grqq

------------------------

\textbf{Skeptizismus bedeutet Zweifel.}\\
Zwei Möglichkeiten des Zweifels:
\begin{itemize}
\item[1.] \textbf{Destruktiv:} Schluss der Allanzweifelbarkeit zu kommen\\
\item[$\rightarrow$] \glqq Lethargie\grqq , als eine logisch \glqq beweisbare\grqq\ Haltung
%TODO Uebernehmen in Presi
\item[2.] \textbf{Konstruktiv:} Versuch der Erweiterung d. Erkenntnis
\item[$\rightarrow$] Erkenntnistheorie nur auf Basis der Logik
\item[$\rightarrow$] Keine Bekämpfung von Mehraussagen, sondern von \glqq unlogischen\grqq Axiomen
\item[$\rightarrow$] Durch radikalen Skept.: Trotzdem kein Erkenntnisgewinn
\end{itemize}

-----------------

\textbf{\emph{Doch} eine Flucht?}\\
\begin{center}
\emph{Dies wäre der Fall, wenn eine Erkenntnistheorie nur an ihrem Erkenntnisgewinn \textbf{an sich} gemessen werden würde!}
\end{center}

---------------------

\textbf{Meta-Sinn:}
\begin{itemize}
\item Abstraktion, pure Logik
\item Bodenlosigkeit (durch anzweifeln des "Bodens") auch als Freiheit
\item[$\Rightarrow$] Ziel != Erkenntnisgewinn
\end{itemize}

----------------------

\textbf{Das \glqq Universum\grqq\ des Skeptizismus}\\
Allzweifel nicht nur: logische Begruendung der Lethargie:\\
- methodischen Zweifel! (Wie Descartes; nur: Abbruch auf letzter Ebene: Weiterfuehrung):\\
Begegnung mit Philosophischer Hydra:\\
Antwort -> zu Zweifel -> mehr "Antworten"; mehr Zweifel\\
Ewigkeit des Zweifels

-----------------

Vor.: Wir \glqq glauben\grqq\ wir denken. wissen aber nicht, was wir eigentlich tun, falls nicht \glqq glauben\grqq \\
\ \\
Fall I Wir denken. Ergo, existieren \glqq wir\grqq . \\
Fall II: Wir denken nicht. Ergo \glqq existiert\grqq\ irgendetwas, was uns glauben macht, zu denken\\
\ \\
Beispielsweise: Ein Programm, Algorithmus, \ldots - eine \glqq Steuerung\grqq\ von \glqq außen\grqq \\

programmierte Entitäten; Nicht-Erkennen der eigenenen Trivialitaet.\\
Unabhaengiges Denken?

-------------------

\begin{center}
Es existiert \glqq etwas\grqq .\\
Darin liegen die \glqq Informationen\grqq\ unseres \glqq Denkens\grqq
\end{center}

------------------

\textbf{Ziel des Skeptizismus folgt aus der Intention des Skeptikers}\\
$\rightarrow$ Skeptizismus bloße Feststellung der Erkenntnislosigkeit\\
\begin{center}
Außerdem ein Mittel der Reflexion, der Rückkehr vom täglichen Dogmatismus
\end{center}

----------------

[Schaubild]
\glqq Was ist jede Erkenntnis, wenn sie nicht auf Logik basiert;\\
was bringt jede Logik, wenn das Leben nicht nach ihr funktioniert.\grqq\
\end{document}



%TODO
 - geschichtlicher Hintergrund / Jahreszahlen
 - evtl. Nihilismus
 - Popper
 - Kritizismus weiter ausbauen