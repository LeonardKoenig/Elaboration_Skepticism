\documentclass[12pt]{beamer}
\usetheme{Berkeley}
\usepackage[utf8]{inputenc}
\usepackage[german]{babel}
\usepackage[T1]{fontenc}
\usepackage{graphicx} %TODO Logo!
\usepackage{csquotes}

\author{Leonard König}
\title{Skeptizismus - Flucht der Philosophie vor der Frage nach der Erkenntnis?}
% personalpronomen? Ja, Nein?

%\setlength{\parskip}{1em}
\usepackage[
	natbib=true,
    backend=biber,
    style=numeric,
    citestyle=numeric,
    sorting=nty,
    sortlocale=de_DE,
]{biblatex}
\addbibresource{text.bib}


\author{Leonard König}
%\title{}
%\setbeamercovered{transparent} 
%\setbeamertemplate{navigation symbols}{} 
%\logo{} 
%\institute{} 
%\date{} 
%\subject{} 
\begin{document}
\begin{frame}
\titlepage
\end{frame}

%\begin{frame}
%\tableofcontents
%\end{frame}


\section{Hinführung zum Thema: Sokrates}
\begin{frame}
Sokrates: Skeptiker oder methodischer Zweifel?
\end{frame}

%\\Einerseits kann man es als eine Devise des sokratischen, methodischen Zweifels auffassen - also der immerwährenden Skepsis, um zur Erkenntnis zu gelangen -, andererseits kann man das Zitat durchaus anders interpretieren: Egal was ich tue, ich kann nie zu einer Erkenntnis kommen, was als Interpretationsansatz mein erster Berührungspunkt mit der philosophischen Haltung des Skeptizismus war.\\
%Mir erschien dieser Schluss immer logisch: Woher soll ich \glqq wissen\grqq? Was ist überhaupt \glqq Wissen\grqq\ oder gar \glqq Erkenntnis\grqq ? Ist \glqq Erkenntnis\grqq\ überhaupt möglich?\\
%Doch zunächst werde ich diese Fragen zurückstellen und stattdessen mich einigen Definitionen widmen.
	\subsection{Definitionen}
		\subsubsection{Skeptizismus}
\begin{frame}{Def.: Skeptizismus}
Duden:\\
\glqq Im weiteren Sinn philosophische Positionen, die Wahrheitsansprüchen gegenüber Verzicht und Zurückhaltung üben und sorgfältige kritische Prüfung verlangen.\\
Im strengeren Sinne Richtungen, die die Beweisbarkeit von Wahrheit (entweder prinzipiell oder partiell) in Zweifel ziehen.\grqq

Beispiel für einen modernen Skeptiker: David Hume
\end{frame}		

\subsubsection{Agnostizismus}
\begin{frame}{Def.: Agnostizismus}
\begin{itemize}
\item geprägt von Thomas Henry Huxley
\item bezog sich vor Allem auf den Glauben
\item[$\Rightarrow$] Unabhängigkeit der Wissenschaft von Religion herstellen
\item Herkunft: $\grave{\alpha}\gamma\nu\bar{\omega}\sigma\iota\varsigma$: \glqq ohne Wissen\grqq\
\item[$\Rightarrow$] vereinzelt synonym zu Skeptizismus verwendet
\end{itemize}
\end{frame}

\subsubsection{Solipsismus}
\begin{frame}{Def.: Solipsismus}
\begin{itemize}
\item skeptische Haltung ggü. allem außer der eigenen Selbst
\item nur die eigene Existenz gesichert
\item oder: Bedeutung der Wahrnehmung abhängig vom Zustand des (denkenden) Ichs %\cite{iep_solipsis} %TODO
\end{itemize}
\end{frame}
\section{Skeptizismus im historischen Kontext}
\subsection{Antike - die Sophisten} %TODO Ordnen, historisch u. nach Position
\begin{frame}{Antike - die Sophisten}
Sophisten waren Wanderlehrer, sie unterrichteten
\begin{itemize}
\item Diskussion
\item Politik
\item Geschichte
\item \emph{Philosophie \& Erkenntnistheorie}
\end{itemize}
Jedoch zeitbedingt viele unterschiedliche Schulen
%TODO Zeitbedingt? Rechtschreibung!
\end{frame}

\begin{frame}{Protagoras und Gorgias}
Protagoras: Mensch das Maß aller Dinge\\% \cite{enc_brit_sophist} %TODO
Gorgias: dogmatischer Skeptizismus:
\begin{itemize}
\item Nichts existiert;
\item Wenn etwas existiert, dann kann man nichts darüber wissen; und
\item Wenn man etwas darüber wissen kann, dann kann dieses nicht mitgeteilt werden.% \cite{iep_on-the-nonex} %TODO
\end{itemize}
\end{frame}

\begin{frame}{Radikaler vs. akademischer Skeptizismus}
\textbf{Akademischer Skeptizismus:}\\
kein Wissen möglich
\textbf{Radikaler Skeptizismus:}\\
Kritische Ansichten gegenüber \emph{jeder} Aussage:\\
\glqq unsere Sinneserfahrungen und Ansichten sind weder wahr, noch falsch\grqq\ %\cite{greek_stough} %TODO
% Der Skeptizismus nach Pyrrho geht nicht so dogmatisch vor, sondern behauptet, man könne über die Aussage selbst, also, dass kein Wissen möglich ist, keine wirkliche Aussage treffen kann (radikaler Skeptizismus). % %TODO http://en.wikipedia.org/wiki/Philosophical_skepticism
\end{frame}

%Der siebte Leiter der antiken Akademie, Arkesilaos, begründete den akademischen Skeptizismus. Er schloss sich dabei seinen Vorgängern an, dass es keine Wahrheit gäbe, weil zu beliebigen (philosophischen) Aussagen gleichberechtigte Gegenargumente oder Positionen zu finden seien. %TODO contradiction!!!
\subsection{Mittelalter - akademischer Skeptizismus versus Kirche}
%TODO Bis hier: 5min
\begin{frame}{Augustinus}
Augustinus:
\begin{itemize}
\item von Ciceros akademischen Skeptizismsus geprägt
\item Römerbriefen: Religiosität und Kritik am Skeptizismus
\item trotzdem Platonische Auslegung der Bibel
\end{itemize}
\glqq Wird jemand darüber zweifeln, dass er lebt, sich erinnert, Einsichten hat, will, denkt, weiß und urteilt? [\ldots] Mag einer auch sonst zweifeln, über was er will, über diese Zweifel selbst kann er nicht zweifeln\grqq\ %\cite{de_trini_x} %TODO
\end{frame}
%TODO Zitat hervorheben, evtl. Manichäismus raus; dafuer mehr Popper
\subsection{Skeptizismus von Descartes bis Kant}
\begin{frame}{Descartes}
Descartes:
\begin{itemize}
\item Übernahm den methodischen Zweifel und  \glqq ich zweifle, also bin ich\grqq
\item Versuch durch reine Logik zu beweisbarer Erkenntnis zu gelangen
\item Dadurch, dass man daran zweifelt, dass man zweifelt, beweise man das eigene Zweifeln
\end{itemize}
\emph{Doch: Weiß man, dass man wirklich zweifelt?}
\end{frame}

\begin{frame}{David Hume}
David Hume:
\begin{itemize}
\item Empiriker: Erkenntnisgewinn nur durch Erfahrung
\item[$\Rightarrow$] es liegt nahe, dass unsere Erfahrung getäuscht ist
\item[$\Rightarrow$] evtl. kein \glqq echter\grqq\ Erkenntnisgewinn möglich
\end{itemize}
$\Rightarrow$ Was ist überhaupt \glqq Erkenntnis\grqq ?
\end{frame}

\begin{frame}{Immanuel Kant}
\begin{itemize}
\item zwar richtige Argumentation, jedoch:\\
Zu einer  \glqq gereiften [\ldots] Urteilskraft\grqq\ gehöre es auch, dass diese \glqq feste und ihrer Allgemeinheit nach bewährte Maximen zum Grunde hat\grqq\ %\cite{kritik} %TODO
\item[$\Rightarrow$] wichtig gegen Dogmatismus und zur Reflexion des Lebens
\item[$\Rightarrow$] jedoch nicht um die eigene Lethargie zu rechtfertigen, nach dem Motto:\\
\glqq Ich kann ja sowieso nichts wissen\grqq
\end{itemize}
\end{frame}

\subsection{Moderner Skeptizismus}
%TODO 10min
%TODO Kritizismus erlaeutern
\begin{frame}{Ethik: Der Nihilismus}
Ethik:\\
\begin{itemize}
\item logisch folgerichtige \glqq Beweisführung\grqq\ für einen Nihilismus
\item jedoch keine Rechtfertigung auf moralischer Basis?
\end{itemize}
Nietzsche:\\
\begin{itemize}
\item Verlust der Metaphysik durch unendlichen Zweifel
\item[$\Rightarrow$] kein Fundament der Moral mehr gegeben
\item[$\Rightarrow$] Nihilismus ist eine Entwertung der obersten Werte der Moral
\item \glqq Umwertung aller Werte\grqq als neue Basis für die Ethik
\end{itemize}
%TODO Verlust der Metaphysik: Nachrecherchieren! Axiome?
\end{frame}

\begin{frame}{Von Nietzsche zu Nelson}
Nelson: \glqq Die Unmöglichkeit der Erkenntnistheorie\grqq
%TODO fuellen
\end{frame}
%TODO Ueberleitung

\begin{frame}{Popper: Kritischer Realismus}
\begin{itemize}
\item Wissenschaftstheorie: Angewandter Skeptizismus:
\item[$\Rightarrow$] Kritik an Induktion etc.
\item andererseits soll über Erfahrung wissenschaftlich gesicherte Erkenntnis gewinnbar sein
\end{itemize}
\glqq Kritischer Realismus\grqq
%TODO %Diese Position nennt man kritischen Realismus.% Rationalismus? sicher? uebepruefen!
\end{frame}

\section{Reflexion - Skeptizismus als Lebenseinstellung?}
%Wir haben jetzt also eine ganze Liste an Fragen abzuarbeiten:
%\begin{itemize}
%\item Wie ist Skeptizismus in diesem Kontext der Fragestellung zu verstehen?\\
%Eventuell sogar: Sind verschiedene Haltungen des Skeptizismus hier differenziert zu betrachten?
%\item Ist der Skeptizismus in sich logisch konsistent?
%\item Hat der Skeptizismus einen epistemologischen Sinn?\\
%Falls nein, ist er eventuell trotzdem sinnvoll in einem abstrakteren Kontext? %konkreter
%% ende der reflexion: aber eine frage habe ich vergessen:
%% der skeptizismus ist erst eine flucht, wenn man ihn benutzt, sonst bleibt er eine moeglichkeit
%% Und da "die Philosophie" nicht "fluechten" kann, ist es eben nur letzteres, maximal
%\end{itemize}
%%
%Wie schon erwähnt, gibt es verschiedene \glqq Arten\grqq\  des Skeptizismus. Um die Leitfrage zu beantworten, macht es zwar Sinn, eine grobe Übersicht über die einzelnen philosophischen Standpunkte zu gewinnen, jedoch ist für die Leitfrage vor Allem der Skeptizismus in seinen erkenntnistheoretischen und metaphysischen Aspekten relevant; das heißt:\\
%Der Fokus liegt auf dem sehr radikalen Skeptizismus von Pyrrho aus der Antike und den daraus hervorgehenden Positionen Humes, Kants und schließlich die des Kritizismus.
%%TODO "Der Fokus" -- unser Fokus!
%
%Einleitend kann man sagen, dass der epistemologische Skeptizismus ein durchaus abstraktes Thema mit sehr \glqq einfachen\grqq\ Argumenten in sich konsistent zu behandeln versucht. Wenn man dies jedoch schon als \glqq Flucht\grqq\ bezeichnet, hat man vorschnell geurteilt.
%
%Der Skeptizismus ist ein Versuch der Philosophie eine Erkenntnistheorie nur auf Basis der Logik aufzubauen. Kein echter Skeptiker würde eine Erkenntnistheorie, die ebenso logisch und ohne Axiome auskommt, ablehnen, selbst wenn sie mehr Aussagen erlaubte. Schon aus diesem Grund hat der Skeptizismus zumindest eine Daseinsberechtigung - wir wollen aber noch mehr:\\
%Wir wollen wissen, ob er auch noch heute eine Sinnhaftigkeit darüber hinaus hat, und eben nicht nur eine \glqq Flucht\grqq\ ist.
%
%Da der Skeptizismus keine Aussage über unsere Erkenntnis liefert, genauer, er liefert die Aussage, dass er keine Aussage darüber geben kann, ist er in diesem Sinne \textbf{nicht} von erkenntniserweiterndem Wert, abseits eben dieser (auch anzweifelbaren) Erkenntnis.\\
%Jedoch soll dies nicht die einzige Beurteilung einer philosophischen Richtung sein. 
%
%Es besteht kein Zweifel darüber, dass der Skeptizismus durchaus eine Berechtigung der Lethargie darstellen kann. Er bietet damit die Grundlage einer \glqq Fluchtmöglichkeit\grqq\ . Schließlich ist er ein Mittel, um mit reiner Logik zu jeder Antwort einen Zweifel zu finden. Dadurch ermöglicht er es aber auch einem, diesen Zweifel als Sokrates'schen Zweifel zu nutzen - wie es auch Descartes in seiner Hinführung zum \glqq cogito ergo sum\grqq\ tat. Nur brach er - bei seinem \glqq letzten Zweifel\grqq , dem Zweifel an dem eigenen Denken - ab. Hätte es Sinn gemacht diesen Zweifel weiterzuführen? Das hängt wohl eindeutig von der Intention ab. Beabsichtigt man wirkliche, weitere Erkenntnis zu gewinnen, so ist wohl eine weitere Fortsetzung schwerlich behilflich, weil man eben gewonnene Erkenntnis wieder eintauscht, um weiteren Zweifel zu erhalten.\\
%Ist man jedoch auf der Suche nach Abstraktion und Phantasie ist es interessant, diesen Weg der Begegnung mit der philosophischen Hydra zu beschreiten. Meint man eine Antwort gefunden zu haben, so eröffnen sich dem Zweifler gleich mehrfache Wege fortzuschreiten und die Ewigkeit des Zweifels zu erkunden.\\
%Es geht dem Skeptiker nicht darum, Beweise führen zu können, es geht ihm darum, den Zweifel zu erkunden, waghalsig Thesen aufzustellen und zu verwerfen.
%
%% evtl. zu viel...
%Noch interessanter wird es jedoch, wenn wir uns überlegen, \emph{was} alles ist was wir eventuell auch nur scheinbar denken. Denn irgendetwas wissen wir ja doch, oder tun wir - nur wissen wir nicht, ob wir dieses Etwas auch \glqq Wissen\grqq , respektive \glqq Tun\grqq , nennen können.\\
%Was ist diese \glqq Überlegung\grqq , dass wir eventuell keine Erkenntnis haben können, wenn nicht Erkenntnis? Informationen?\\
%Letztendlich gibt es also diese binären Möglichkeiten: Wir denken -oder nicht.
%Falls wir denken, \glqq existiert\grqq\ dieses Denken. Falls nicht, was ist es dann? Es wird \glqq uns\grqq\ vorgegaukelt, wir dächten. Ergo \glqq denken\grqq\  wir nicht, vielmehr \glqq wird für uns gedacht\grqq . 
%%TODO evtl. "Programm" klairifizieren: Steuerung; "Abschied vom Subjekt"; "Makrokosmos"; Variierung von "programmiert"
%Beispielsweise ist es vorstellbar, dass wir eigentlich nur programmierte Entitäten sind - programmiert, sodass sie die eigene Trivialität nicht erkennen und sich als \glqq viel zu komplex\grqq\ einstufen, um programmiert zu sein. Diese Entitäten \glqq denken\grqq\ also nur Prozesse, die eigentlich lediglich durch das Programm vorgegeben sind. Von unabhängigen Denken können wir nicht sprechen, aber eher vom Ablaufen eines Algorithmus'.\\ Andererseits, damit diese Entitäten zuerst einmal \glqq denken zu denken\grqq , muss eben dieses Übergeordnete existieren, ein Programm, in dem das \glqq Denken\grqq\ festgeschrieben ist. Wir kommen also zum Schluss, unsere Gedanken \glqq existieren\grqq\  - wenn auch nicht unbedingt in unserem Kopf. Weiter hieße das, dass \glqq etwas\grqq\ existiert, dieses \glqq etwas\grqq , kann jedoch auch rein informationell sein, im Sinne von nicht-substanziell, wie es beispielsweise auch der Begriff \glqq digital\grqq\ charakterisiert.
%%TODO verbessern; zur Not raus
%
%% Ende:
%Wenn man weiter prüft, muss man eingestehen, dass der Skeptizismus in \textbf{der} Hinsicht von Nutzen ist, als dass er einen mit dem Mittel ausstattet, aktiv vom täglichen praktischen Dogmatismus zurückzukehren und alles zu hinterfragen.
%%\subsection{Schluss}
%%\printbibliography %TODO quickbuild: bib}
%
%
%

\end{document}