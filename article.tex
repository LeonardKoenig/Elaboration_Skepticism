\documentclass[12pt,a4paper,final]{article}
\usepackage[utf8]{inputenc}
\usepackage[ngerman]{babel}
\usepackage{amsmath}
\usepackage{amsfonts}
\usepackage{amssymb}
\linespread{1.5}
%\usepackage[left=2cm,right=2cm,top=2cm,bottom=2cm]{geometry}
%line bug (Praesentationsprogramme)
\usepackage{newtxtext}
\usepackage[left=4cm,right=2cm,top=3cm,bottom=3cm]{geometry}
\begin{document}
Mit meiner Arbeit werde ich untersuchen, inwiefern der Skeptizismus als Flucht der Philosophie vor der Frage nach der Erkenntnis verstanden werden kann. Dies geschieht unter Erläuterung der philosophie-geschichtlichen Aspekte, wie der Entwicklung des Skeptizismus von der Antike bis zur Neuzeit, sowie der sich in diesem Prozess veränderte Zielrichtung. Dies werde ich mit einer %Latex; modifizierung statt aenderung
 -Präsentation unterstützen. Zu Beginn erfolgt eine Vorstellung der historischen Wandlungen des Skeptizismus. In einer sich im ständig im Umbruch befindenden Gesellschaft, hatte eine skeptisch-kritische Haltung schon immer eine wichtige Rolle gespielt, und wird diese weiterhin einnehmen.\\
 Abschließend werde ich einem Fazit zu der Leitfrage Stellung beziehen und den Versuch machen, diese zu beantworten.
 
 Der sich immer wieder weiterentwickelnde Skeptizismus, hat gerade in der aktuellen Zeit im wissenschaftlichen Bereich eine starke Resonanz.\\ %uebrleitung
Einerseits konnte ich die Kritik an der skeptischen Haltung schon immer verstehen, andererseits, finde ich, dass der Skeptizismus mehr darstellt, als die Kritiker sehen wollen. %ueberarbeiten
Darüber mich in zahlreichen Diskussionen mit anderen ausgetauscht. Im Gegensatz zu anderen philosophischen Ansätzen (wie z.B. dem Empirismus %todo
) ist der Skeptizismus auf Grund seiner seiner bestechenden Logik in sich sehr geschlossen. Die Frage nach dem Zweck und der Sinnhaftigkeit des Skeptizismus baut eine Brücke zwischen Metaphysik und Erkenntnistheorie auf, und hinterfragt grundlegend die Sinnhaftigkeit mancher philosophischer Überlegungen.%Ueberarbeiten
Ich überprüfe, ob man solche hoch abstrakten philosophischen Thematiken auch außerhalb ihres Selbstzweckes begründen kann.%Sinn geben
Im Philosophie-Unterricht ist gerade zum Thema Metaphysik immer diese Frage nach der Sinnhaftigkeit über solche Problematiken zu philosophieren und zu diskutieren aufgekommen.%Gegenstand spaeter
Dabei ist mir klar geworden, dass der Zweck über solche Themen zu diskutieren nicht darin liegt, Erkenntnis über sie selber zu gewinnen, sondern in vielerlei Hinsicht ein höheres philosophisches Verständnis zu gewinnen, sowie den Blick auf andere, konkretere Fragen der Philosophie zu stärken. Zudem hat die Metaphysik - so sehr abstrakt und ohne Grundlage sie auch ist - in unser aller Leben eine wichtige, sinngebende Rolle.

Die Verwendung von LaTeX eröffnet mir mehr Möglichkeiten als herkömmliche \\
Präsentationsprogramme, sowie eine einfachere, transparentere und geordnetere Arbeitsweise.\\
Dies eröffnet mir eine sehr schlicht-übersichtliche Darstellung meiner Ergebnisse.\\

Während die geschichtsbezogene Recherche sich als relativ einfach herausstellte, die Verinnerlichung der sehr anspruchsvollen Literatur (es sei bspw. nur die Kritik der reinen Vernunft %Quelle
 zu nennen) hat mich einige Zeit gekostet, mich aber auch herausgefordert. Ich habe versucht mich vorwiegend auf Primärquellen zu beziehen. Aufgrund der schwierigen Verfügbarkeit von vor allem antik-historischen Quellen, habe ich oft auf digitalisierte Versionen der Originale zurückgegriffen.
Meine Nachforschungen waren davon geprägt, dass ich sehr oft den Verweisen einzelner Autoren nachgegangen bin und mir dadurch einen guten Überblick über die Thematik erschließen konnte. Weitere Hilfsmittel zur Verinnerlichung der Problematik benötigte ich nicht, da ich mich fortgesetzt mit der Frage beschäftigt und die vorigen Erkenntnisse immer wieder hinterfragt habe.\\

Aufgrund der Tatsache, dass ich mir relativ früh meiner Themenwahl bewusst, und ich mich auch schon weitreichend informiert war und beschäftigt habe, fiel mir - nach anfänglichen Schwierigkeiten - die Präzisierung der Themenwahl nicht allzu schwer, sodass in zwei kurzen Beratungsgesprächen mit meinem betreuenden Fachlehrer beendet werden konnte. Diese waren mir bei der weiterführenden Ausarbeitung sehr dienlich.

Auf wenn ich mit Gruppenarbeit gute Erfahrung gemacht habe, wie bei meiner MSA-Prüfung, habe ich diesmal entschlossen, dieses weitreichende Thema allein zu bearbeiten.
\end{document}