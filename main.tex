\documentclass[12pt,a4paper]{article}
\usepackage[utf8]{inputenc}
\usepackage[ngerman]{babel}
\usepackage[T1]{fontenc}
\usepackage{amsmath}
\usepackage{amsfonts}
\usepackage{amssymb}
\usepackage{graphicx}
%\linespread{1.5}
\usepackage[left=2cm,right=4cm,top=3cm,bottom=3cm]{geometry}
%\usepackage[left=2cm,right=8cm,top=3cm,bottom=3cm]{geometry}
\author{Leonard König}
\title{Skeptizismus - Flucht der Philosophie vor der Frage nach der Erkenntnis?}
% personalpronomen? Ja, Nein?
\begin{document}
\maketitle
%\newpage
\tableofcontents
\newpage
\section{Hinführung zum Thema: Sokrates}
\glqq Ich weiß, dass ich nichts weiß\grqq. Es gibt verschiedene Auslegungen dieses wohl berühmtesten Zitats Sokrates'.% Sokrates'?
\\Einerseits kann man es als eine Devise des sokratischen, methodischen Zweifels auffassen - also der immerwährenden Skepsis, um zur Erkenntnis zu gelangen - , andererseits kann man dieses Zitat durchaus anders interpretieren: Egal was ich tue, ich kann nie zu einer Erkenntnis kommen. Dieser skeptizistische Interpretationsansatz war mein erster Berührungspunkt mit dieser philosophischen Haltung.\\
Mir erschien dieser Schluss immer logisch: Woher soll ich \glqq wissen\grqq? Was ist überhaupt \glqq Wissen\grqq\ oder gar \glqq Erkenntnis\grqq\ ? Ist \glqq Erkenntnis\grqq\ überhaupt möglich?\\
Doch zunächst werde ich diese Fragen zurückstellen und stattdessen einige Definitionen widmen.
	\subsection{Definitionen}
		\subsubsection{Skeptizismus}
Der Duden sagt über den Skeptizismus:\\
\glqq Im weiteren Sinn philosophische Positionen, die Wahrheitsansprüchen gegenüber Verzicht und Zurückhaltung üben und sorgfältige kritische Prüfung verlangen.\\
Im strengeren Sinne Richtungen, die die Beweisbarkeit von Wahrheit (entweder prinzipiell oder partiell) in Zweifel ziehen.\grqq% Zitat checken, cite nutzen etc.
\\Einer der moderneren Hauptvertreter des Skeptizismus ist
%unter anderen?
der Philosoph Hume.\\%Quelle
		\subsubsection{Agnostizismus}
Der Begriff \glqq Agnostizismus\grqq\ wurde vom Biologen Thomas Henry Huxley geprägt. Agnostizismus ist oft als eine Unterposition des Skeptizismus aufzufassen, in der man davon ausgeht, dass über jegliche oder nur bestimmte überirdische% irden/irdlich/mystisch?
\ Ideen - wie beispielsweise einem Gott oder der Wiedergeburt - keine Erkenntnis zu gewinnen ist. In der Regel fasst man den Begriff noch enger um eine Religion, sprich als Abgrenzung zu The- und Atheisten.\\ 
Vereinzelt jedoch wird der Begriff Agnostizismus synonym zum Skeptizismus verwendet. Dabei geht man davon aus, dass, eytomologisch gesehen, Agnostizismus \glqq ohne Wissen\grqq\ bedeutet.
		\subsubsection{Solipsismus}
Der Solipsismus ist eine philosophische Schule, die Allem außer meiner Selbst gegenüber eine skeptische Haltung einnimmt. Solche Philosophen vertreten die Meinung, dass entweder nur ihre eigene Existenz gesichert ist, oder, dass die Bedeutung der Wahrnehmung vom Zustand des eigenen (denkenden) Ichs abhängt.% http://www.iep.utm.edu/solipsis/ ; wiki.de
\section{Skeptizismus im historischen Kontext}
	\subsection{Antike - die Sophisten}
Die Sophisten waren Wanderlehrer, die Diskussion und antike Formen der Philosophie lehrten. Darunter interessiert natürlich vor allem die Epistemologie/Erkenntnistheorie der Antike. Sie haben jedoch logischerweise nicht unbedingt ähnliche Ansichten, was diese Richtung der Philosophie betrifft. Während Protagoras den Menschen für das Maß aller Dinge hält% http://en.wikipedia.org/wiki/Sophism
% Begründung logischerweise
, ist Gorgias:
\begin{itemize}
\item Nichts existiert;
\item Wenn etwas existiert, dann kann man nichts darüber wissen; und
\item Wenn man etwas darüber wissen kann, dann kann dieses nicht mitgeteilt werden.
\item Wenn es mitgeteilt werden kann, kann es nicht verstanden werden.
\end{itemize}% frei aus dem engl.: http://en.wikipedia.org/wiki/Gorgias#On_the_Non-Existent
Gorgias kann man mit diesen Aussagen einem dogmatischen Skeptizismus zuordnen, denn radikale Skeptiker würden selbst der letzten Aussage noch skeptisch gegenüber stehen.\\
Pyrrho beispielsweise meinte: \glqq unsere Sinneserfahrungen und Ansichten sind weder wahr, noch falsch\grqq.% frei a.d. engl.
% Greek Skepticism: A Study in Epistemology
% Charlotte Stough
% ISBN: 9780520016040
% https://books.google.de/books?id=ky_z4NHa-oUC&pg=PA19&lpg=PA19&dq=%22are+in+fact+neither+true+nor+false.%22+pyrrho&source=bl&ots=y-7F1B3pkD&sig=FCab_nkze1lRNhpW20KFCrK0ngk&hl=en&sa=X&ei=0BGgVMnIBMb_UPfFgsgL&ved=0CCcQ6AEwAQ#v=onepage&q=%22are%20in%20fact%20neither%20true%20nor%20false.%22%20pyrrho&f=false
 Auch wenn der Pyrrhonismus das Ziel der Ataraxia, der mentalen Unerschütterlichkeit hatte und zu diesem Zweck einer skeptizistischen Haltung folgte, ist er einer der bekannteren Skeptiker der Antike.% http://en.wikipedia.org/wiki/Pyrrho
% begin einarbeitung
\\-----\\
Es gibt jedoch einige Strömungen des Skeptizismus. So gibt es den akademischen Skeptizismus, der die Aussage trifft, das Wissen nicht möglich ist. Der Skeptizismus nach Pyrrho geht nicht so dogmatisch vor, was zugespitzt - im radikalen Skeptizismus - dazu führt, dass man selbst über die Aussage selbst, also, dass kein Wissen möglich ist, keine wirkliche Aussage treffen kann.% http://en.wikipedia.org/wiki/Philosophical_skepticism ; umformulieren!
\\-----% end einarbeitung  
\\Der siebte Leiter der antiken Akademie, Arkesilaos, begründete den akademischen Skeptizismus. Er schloss sich dabei seinen Vorgängern an, dass zu beliebigen (philosophischen) Aussagen gleichberechtigte Gegenargumente oder Positionen zu finden seien.
	\subsection{Mittelalter - akademischer Skeptizismus versus Kirche}
Diese Richtung des Skeptizismus herrschte noch einige Zeit an und prägte sogar den Kirchenlehrer Augustinus. Schließlich wurde sein Interesse an der Philosophie von Cicero geweckt, einem Anhänger des akademischen Skeptizismus zu dieser Zeit. Er selber vertrat sogar diese Philosophie und hat sie benutzt um gegen den Manichäismus zu argumentieren, einer Offenbarungsreligion, den er vor seiner skeptischen Zeit selber angehörte.\\
Später jedoch wurde er zu einem der größten Kritiker seiner ehemaligen Schule. Zwar bezog er sich - wie auch die Akademiker - auf Platon, jedoch legt er die Bibel danach aus, was in einer sehr religiösen Philosophie mündete, nachdem er den \glqq Römerbrief\grqq\ des Paulus gelesen hatte.\\
Dennoch beschäftigt sich Augustinus weiterhin mit der Wahrheit und nimmt dabei Descartes \glqq cogito ergo sum\grqq\ voraus:\\
„wird jemand darüber zweifeln, dass er lebt, sich erinnert, Einsichten hat, will, denkt, weiß und urteilt? […] Mag einer auch sonst zweifeln, über was er will, über diese Zweifel selbst kann er nicht zweifeln“
% – De trinitate X, 10 http://de.wikipedia.org/wiki/Augustinus_von_Hippo
	\subsection{Skeptizismus von Descartes bis Kant}
		\paragraph{Descartes, Locke, Hume, Kant}
Descartes hat zwei sehr bedeutende Dinge von seinen Vorgängern (eventuell unwissend) übernommen: Den methodischen Zweifel (von Sokrates) und, wie gerade erwähnt, den Ausspruch \glqq ich zweifle, also bin ich\grqq.\\
Descartes hat damit versucht, ohne jede Voraussetzung von einer Wahrheit zumindest \emph{irgendetwas} zu \glqq beweisen\grqq, etwas was über allen Zweifel erhaben ist.\\
Er meinte, man könne zweifeln so viel man möge, an allem und jedem, aber, genau das zeige, man zweifele. Also existiert man.\\
Doch: Weiß man, dass man wirklich zweifelt?\\

Gegner des Rationalisten Descartes waren die Empiristen. Empirische Schulen gab es auch schon in der Antike (unter anderem bei Aristoteles), der britische Empirismus ist hier jedoch interessanter, speziell David Hume.\\ % Begruendung
 In seinem ersten großen Werk \glqq A Treatise of Human Nature\glqq behandelt er unter anderem die Probleme der Induktion aber auch die der Erkenntnis. Welche \glqq Arten\grqq\ von Erkenntnis gibt es? Welchen können wir \grqq\ trauen\grqq?\\

Eine Frage, die auch Kant beschäftigt hat. Er selber schreibt Hume seine \glqq Befreiung\grqq\ vom Dogmatismus zu. Kant gibt zu, dass Humes Folgerung allerdings richtig ist, jedoch soll man es dabei nicht belassen, sondern zu einer  \glqq gereiften [...] Urteilskraft\grqq\ gehört auch, dass diese \glqq feste und ihrer Allgemeinheit nach bewährte Maximen zum Grunde hat\grqq.% http://gutenberg.spiegel.de/buch/kritik-der-reinen-vernunft-1-auflage-3508/121
Letztendlich also keine Kritik am radikalen Skeptizismus, sondern nur an der reinen Ausübung desselben, an der Lethargie.\\% berichtigen!
	\subsection{Moderner Skeptizismus}
		%Popper, Nietzsche, Schopenhauer
		%Stichwort: Neukantianismus
\section{Reflexion - Skeptizismus als Lebenseinstellung?}
	\subsection{Schluss}
%Schaubild 

 %\glqq Was ist jede Erkenntnis, wenn sie nicht auf Logik basiert;\\
 %was bringt jede Logik, wenn das Leben nicht nach ihr funktioniert.\grqq

%-------------\\
%Übergangsliste:
%Antike -(Akademie)-> MittelA. -(Augustinus \glqq cogito ergo sum\grqq\ )-> Rat.+Emp. -(Kant)-> Moderne
\section*{Quellen}
\end{document}


---------------
490 BC -  420 BC Protagoras (-> Sophisten)
470/469 BC - 399 BC Sokrates (-> Sophisten)
360 BC - 270 BC Pyrrho (?)
266 BC - 90 BC  Arcesilaus - akademie


 Socrates had said, \glqq This alone I know, that I know nothing.\grqq\ But Arcesilaus went farther and denied the possibility of even the Socratic minimum of certainty: \glqq I cannot know even whether I know or not.\grqq
(One or more of the preceding sentences incorporates text from a publication now in the public domain: Chisholm, Hugh, ed. (1911). \glqq Academy, Greek\grqq. Encyclopædia Britannica (11th ed.). Cambridge University Press.
\text{http://en.wikisource.org/wiki/1911_Encyclop%C3%A6dia_Britannica/Academy,_Greek}
)
