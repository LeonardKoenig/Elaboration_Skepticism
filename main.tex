\documentclass[12pt,a4paper]{article}
\usepackage[utf8]{inputenc}
\usepackage[ngerman]{babel}
\usepackage[T1]{fontenc}
\usepackage{amsmath}
\usepackage{amsfonts}
\usepackage{amssymb}
\usepackage{graphicx}
\linespread{1.5}
%\usepackage[left=2cm,right=4cm,top=3cm,bottom=3cm]{geometry}
\usepackage[left=2cm,right=8cm,top=3cm,bottom=3cm]{geometry}
<<<<<<< HEAD
\author{Leonard König}
=======
\a1uthor{Leonard König}
>>>>>>> correction
\title{Skeptizismus - Flucht der Philosophie vor der Frage nach der Erkenntnis?}
% personalpronomen? Ja, Nein?
\begin{document}
\maketitle
%\newpage
\tableofcontents
\newpage
\section{Hinführung zum Thema: Sokrates}
\glqq Ich weiß, dass ich nichts weiß\grqq. Es gibt verschiedene Auslegungen dieses wohl berühmtesten Zitats Sokrates'.% Sokrates'?
\\Einerseits kann man es als eine Devise des sokratischen, methodischen Zweifels auffassen - also der immerwährenden Skepsis, um zur Erkenntnis zu gelangen - , andererseits kann man das Zitat durchaus anders interpretieren: Egal was ich tue, ich kann nie zu einer Erkenntnis kommen, was als Interpretationsansatz mein erster Berührungspunkt mit der philosophischen Haltung des Skeptizismus war.\\
Mir erschien dieser Schluss immer logisch: Woher soll ich \glqq wissen\grqq? Was ist überhaupt \glqq Wissen\grqq\ oder gar \glqq Erkenntnis\grqq\ ? Ist \glqq Erkenntnis\grqq\ überhaupt möglich?\\
Doch zunächst werde ich diese Fragen zurückstellen und stattdessen mich einigen Definitionen widmen.
	\subsection{Definitionen}
		\subsubsection{Skeptizismus}
Der Duden sagt über den Skeptizismus:\\
\glqq Im weiteren Sinn philosophische Positionen, die Wahrheitsansprüchen gegenüber Verzicht und Zurückhaltung üben und sorgfältige kritische Prüfung verlangen.\\
Im strengeren Sinne Richtungen, die die Beweisbarkeit von Wahrheit (entweder prinzipiell oder partiell) in Zweifel ziehen.\grqq% Zitat checken, cite nutzen etc.
\\Einer der moderneren Hauptvertreter des Skeptizismus ist
%unter anderen?
der Philosoph Hume.\\%Quelle
		\subsubsection{Agnostizismus}
Der Begriff \glqq Agnostizismus\grqq\ wurde vom Biologen Thomas Henry Huxley geprägt. Agnostizismus ist oft als eine Unterposition des Skeptizismus aufzufassen, in der man davon ausgeht, dass über jegliche oder nur bestimmte überirdische% irden/irdlich/mystisch?
\ Ideen - wie beispielsweise einem Gott oder der Wiedergeburt - keine Erkenntnis zu gewinnen ist. In der Regel fasst man den Begriff noch enger um eine Religion, sprich als Abgrenzung zu The- und Atheisten.\\ 
Vereinzelt jedoch wird der Begriff Agnostizismus synonym zum Skeptizismus verwendet. Dabei geht man davon aus, dass, eytomologisch gesehen, Agnostizismus \glqq ohne Wissen\grqq\ bedeutet.
		\subsubsection{Solipsismus}
Der Solipsismus ist eine philosophische Schule, die Allem außer meiner Selbst gegenüber eine skeptische Haltung einnimmt. Solche Philosophen vertreten die Meinung, dass entweder nur ihre eigene Existenz gesichert ist, oder, dass die Bedeutung der Wahrnehmung vom Zustand des eigenen (denkenden) Ichs abhängt.% http://www.iep.utm.edu/solipsis/ ; wiki.de
\section{Skeptizismus im historischen Kontext}
	\subsection{Antike - die Sophisten}
Ein kurzer Überblick: Die Sophisten waren Wanderlehrer, die Diskussion und antike Formen der Philosophie lehrten. Hierbei interessiert natürlich vor allem die Epistemologie/Erkenntnistheorie der Antike. Sie haben jedoch logischerweise nicht unbedingt ähnliche Ansichten, was diese Richtung der Philosophie betrifft, schließlich waren die Schulen teilweise um Jahrhunderte entfernt. Während Protagoras den Menschen für das Maß aller Dinge hält% http://en.wikipedia.org/wiki/Sophism
% Begründung logischerweise <- fixed?
, ist Gorgias einem dogmatischen Skeptizismus zuzuordnen:
\begin{itemize}
\item Nichts existiert;
\item Wenn etwas existiert, dann kann man nichts darüber wissen; und
\item Wenn man etwas darüber wissen kann, dann kann dieses nicht mitgeteilt werden.
\item Wenn es mitgeteilt werden kann, kann es nicht verstanden werden.
\end{itemize}% frei aus dem engl.: http://en.wikipedia.org/wiki/Gorgias#On_the_Non-Existent
Radikale Skeptiker würden selbst der letzten Aussage noch skeptisch gegenüber stehen.\\
Pyrrho beispielsweise meinte: \glqq unsere Sinneserfahrungen und Ansichten sind weder wahr, noch falsch\grqq.% frei a.d. engl.
% Greek Skepticism: A Study in Epistemology
% Charlotte Stough
% ISBN: 9780520016040
% https://books.google.de/books?id=ky_z4NHa-oUC&pg=PA19&lpg=PA19&dq=%22are+in+fact+neither+true+nor+false.%22+pyrrho&source=bl&ots=y-7F1B3pkD&sig=FCab_nkze1lRNhpW20KFCrK0ngk&hl=en&sa=X&ei=0BGgVMnIBMb_UPfFgsgL&ved=0CCcQ6AEwAQ#v=onepage&q=%22are%20in%20fact%20neither%20true%20nor%20false.%22%20pyrrho&f=false
Es gibt jedoch einige Strömungen des Skeptizismus. So gibt es den akademischen Skeptizismus, der die Aussage trifft, dass Wissen nicht möglich ist. Der Skeptizismus nach Pyrrho geht nicht so dogmatisch vor, sondern behauptet, man könne über die Aussage selbst, also, dass kein Wissen möglich ist, keine wirkliche Aussage treffen kann (radikaler Skeptizismus).% http://en.wikipedia.org/wiki/Philosophical_skepticism
Eigentlich verfolgte Pyrrho jedoch das Ziel der Ataraxia, der mentalen Unerschütterlichkeit.\\
% http://en.wikipedia.org/wiki/Pyrrho ; Ataraxia <=> Lethargie

Der siebte Leiter der antiken Akademie, Arkesilaos, begründete den akademischen Skeptizismus. Er schloss sich dabei seinen Vorgängern an, dass es keine Wahrheit gäbe, weil zu beliebigen (philosophischen) Aussagen gleichberechtigte Gegenargumente oder Positionen zu finden seien.
	\subsection{Mittelalter - akademischer Skeptizismus versus Kirche}
Diese Richtung des Skeptizismus dauerte noch einige Zeit an und prägte sogar den Kirchenlehrer Augustinus. Schließlich wurde sein Interesse an der Philosophie durch Ciceros Werke geweckt, einem Anhänger des akademischen Skeptizismus im ersten Jahrhundert vor Christus. Er selber vertrat sogar diese Philosophie und hat sie benutzt, um gegen den Manichäismus zu argumentieren, einer Offenbarungsreligion, der er vor seiner skeptischen Zeit selber angehörte.\\% evtl umformulieren
Später jedoch wurde er zu einem der größten Kritiker seiner ehemaligen Schule. Zwar bezog er sich - wie auch die Akademiker - auf Platon, jedoch legt er die Bibel im Sinne der platonischen Philosophie aus, was in einer sehr religiösen Richtung mündete.\\% , nachdem er den \glqq Römerbrief\grqq\ des Paulus gelesen hatte.\\
Dennoch beschäftigte sich Augustinus weiterhin mit der Wahrheit und nahm dabei Descartes \glqq cogito ergo sum\grqq\ voraus:\\
„Wird jemand darüber zweifeln, dass er lebt, sich erinnert, Einsichten hat, will, denkt, weiß und urteilt? […] Mag einer auch sonst zweifeln, über was er will, über diese Zweifel selbst kann er nicht zweifeln“.
% – De trinitate X, 10 http://de.wikipedia.org/wiki/Augustinus_von_Hippo
	\subsection{Skeptizismus von Descartes bis Kant}
		\paragraph{Descartes, Locke, Hume, Kant}
Descartes hat zwei sehr bedeutende Dinge von seinen Vorgängern (eventuell unwissend) übernommen: Den methodischen Zweifel (von Sokrates) und, wie gerade erwähnt, den Ausspruch \glqq ich zweifle, also bin ich\grqq.\\% -- "(eventuell unwissend)" ?
Descartes hat damit versucht, ohne jede Voraussetzung von einer Wahrheit zumindest \emph{irgendetwas} zu \glqq beweisen\grqq, etwas was über allen Zweifel erhaben ist.\\
Er meinte, man könne zweifeln so viel man möge, an allem und jedem, aber, genau das zeige, man zweifele. Also existiere man.\\ % -- möge ++wolle ?
\emph{Doch: Weiß man, dass man wirklich zweifelt?}\\ % evtl. als Überleitung nutzen, sonst verlaengern. Einbetten: "Ich bezweifle das."

Gegner des Rationalisten Descartes waren die Empiristen. Empirismus ist die Lehre davon, dass Erkenntnis (nur) durch Erfahrung gewonnen werden kann - im Gegensatz zum Rationalismus, der lehrt, dass man angeborene Ideen hat, die nur bestätigt werden müssen. Deshalb ist der Empirismus oft auch "näher" am Skeptizismus, schließlich liegt es nahe, dass, wenn man nur durch Wahrnehmung zu Erkenntnis gelangt, diese Wahrnehmung verfälscht ist und man damit zu keiner "echten" Erkenntnis kommt. David Hume, ein britischer Empirist war genau dieser Auffassung.\\
Diese Probleme, von den Arten der Erkenntnis und ob es überhaupt eine Erkenntnis gibt, sowie über die Problematik der Induktion behandelt er in seinem ersten großen Werk \glqq A Treatise of Human Nature\grqq.

Fragen, die auch Kant beschäftigt haben. Er selber schrieb Hume seine \glqq Befreiung\grqq\ vom Dogmatismus zu. Kant gab zu, dass Humes skeptische Folgerungen% erwaehnen von Folgerungen
\ zwar richtig seien, jedoch solle man es dabei nicht belassen, sondern, so meint Kant, zu einer  \glqq gereiften [...] Urteilskraft\grqq\ gehöre es auch, dass diese \glqq feste und ihrer Allgemeinheit nach bewährte Maximen zum Grunde hat\grqq.% http://gutenberg.spiegel.de/buch/kritik-der-reinen-vernunft-1-auflage-3508/121

Was Kant damit meinte: Der Skeptizismus sei zwar ein wichtiges Werkzeug, da man mit ihm das eigene Leben skeptisch hinterfragen und reflektieren könne; man solle aber den Skeptizismus nicht als philosophische Position ansehen, mit der man die eigene Lethargie rechtfertige, indem man resigniert schließe: \glqq Ich kann ja sowieso nichts wissen\grqq .
% evtl. besser formulieren.

	\subsection{Moderner Skeptizismus}
<<<<<<< HEAD
Das ist eigentlich ein ethisches Problem: Warum sollte ich nicht meine nihilistische Lethargie nicht über den Skeptizismus begründen dürfen? Ist es möglich die Lethargie nicht nur zu begründen, sondern auch zu rechtfertigen? Schließlich: Ich kann auch nicht wissen, ob es nicht \emph{doch} eine erkennbare Welt gibt. Schuldete ich nicht dieser meinen Beitrag, meine Mithilfe?\\ % überleitung zu nietzsche und nelson -> ethik / politologie bzw. gesellschaft

		
		
		
		
=======
Mit dieser Argumentation betritt man jedoch ethisches Terrain, denn: Warum sollte ich nicht meine nihilistische Lethargie nicht über den Skeptizismus begründen dürfen? Ist es möglich die Lethargie nicht nur zu begründen, sondern auch zu rechtfertigen? Schließlich: Ich kann auch nicht wissen, ob es nicht \emph{doch} eine erkennbare Welt gibt. Schuldete ich nicht dieser meinen Beitrag, meine Mithilfe?\\
An diesem Punkt setzt Nietzsche an. Er hält so eine Argumentation für logisch valide, moralisch jedoch, für verwerflich.\\% überleitung zu nietzsche und nelson -> ethik / politologie bzw. gesellschaft
Der Nihilismus bedeutet für Nietzsche, die obersten Werte der Moral zu entwerten, da durch den Verlust der Metaphysik kein Fundament der Moral mehr gegeben ist. Aus diesem Grund versucht er mit der "Umwertung aller Werte" eine neue Basis für die Ethik zu schaffen, in deren Rahmen er den Nihilismus verurteilt. Auf diese ethische Argumentation gehe ich jedoch im Verlauf dieses Vortrages nicht weiter ein.\\
Ein weiterer Vertreter der Position, dass Ethik und Skeptizismus mit einander verbunden sind, ist Leonard Nelson.\\
Wie die Pyrrhoniker versuchte er in seinem Hauptwerk "Die Unmöglichkeit der Erkenntnistheorie", selbiges zu beweisen.
% berguendung der ausführlichkeit der behandlung von leonard nelson
% ---
% Popper:
In seiner Wissenschaftstheorie kritisiert Popper den Einsatz der Induktion als wissenschaftliches Mittel zum Erkenntnisgewinn. Ursprünglich vertritt er damit also eine skeptizistische Position, versucht diese jedoch dann zu überwinden.\\
Er spricht sich dafür aus, dass auf anderem Wege gesicherte Erkenntnisse gewonnen werden können.\\
Diese Position nennt man kritischen Realismus.% Rationalismus? sicher? uebepruefen!


>>>>>>> correction
		% Popper, Nietzsche, Schopenhauer
		% Stichwort: Neukantianismus
		% Nelson; Kritizsimus
\section{Reflexion - Skeptizismus als Lebenseinstellung?}
	\subsection{Schluss}
%Schaubild 

 %\glqq Was ist jede Erkenntnis, wenn sie nicht auf Logik basiert;\\
 %was bringt jede Logik, wenn das Leben nicht nach ihr funktioniert.\grqq

%-------------\\
%Übergangsliste:
%Antike -(Akademie)-> MittelA. -(Augustinus \glqq cogito ergo sum\grqq\ )-> Rat.+Emp. -(Kant)-> Moderne
\section*{Quellen}
\end{document}


---------------
490 BC -  420 BC Protagoras (-> Sophisten)
470/469 BC - 399 BC Sokrates (-> Sophisten)
360 BC - 270 BC Pyrrho (?)
266 BC - 90 BC  Arcesilaus - akademie


 Socrates had said, \glqq This alone I know, that I know nothing.\grqq\ But Arcesilaus went farther and denied the possibility of even the Socratic minimum of certainty: \glqq I cannot know even whether I know or not.\grqq
(One or more of the preceding sentences incorporates text from a publication now in the public domain: Chisholm, Hugh, ed. (1911). \glqq Academy, Greek\grqq. Encyclopædia Britannica (11th ed.). Cambridge University Press.
\text{http://en.wikisource.org/wiki/1911_Encyclop%C3%A6dia_Britannica/Academy,_Greek}
)


Anmerkung: evtl. Zeitdaten
